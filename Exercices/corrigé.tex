\documentclass[12pt,oneside,french]{article}

\usepackage[T1]{fontenc}
\usepackage[utf8]{inputenc}
\usepackage[a4paper]{geometry}
    \geometry{verbose,tmargin=2cm,bmargin=2cm,lmargin=1.5cm,rmargin=1.5cm}
\usepackage{lmodern}
    \renewcommand{\familydefault}{\sfdefault}
\usepackage[french]{babel}
    \DecimalMathComma
    \renewcommand{\FrenchLabelItem}{\textbullet}
    \parindent=0cm
\usepackage{multicol}
    \setlength{\columnsep}{0.5cm}
    \pagestyle{empty}
\usepackage{enumerate}
\usepackage[table]{xcolor}
\usepackage{tikz}
\usepackage{amsthm}
    \theoremstyle{definition}
        \newtheorem{exo}{Exercice}
\usepackage{tabularx}
    \renewcommand{\arraystretch}{1.4}
\usepackage{hyperref}
    \hypersetup{
        colorlinks=true,
        linkcolor=black,
        filecolor=black,      
        urlcolor=blue,
    }
\usepackage{scratch3}
    \setscratch{baseline=c, scale=0.88}
\usepackage{listings}
    \lstset{
        language=Python,
        showstringspaces=false,
        basicstyle=\ttfamily,
        keywordstyle=\bfseries\color{violet},
        commentstyle=\itshape\color{black!40},
        identifierstyle=\color{black},
        numberstyle=\color{green!40!black},
        inputencoding=utf8,
        extendedchars=true,
        breaklines,
        literate=
            {à}{{\`a}}1
            {è}{{\`e}}1
            {é}{{\'e}}1
            {ê}{{\^e}}1
            {ë}{{\"e}}1
            {â}{{\^a}}1
            {ö}{{\"o}}1
            {ô}{{\^o}}1
            {î}{{\^i}}1
            {û}{{\^u}}1
            {ç}{{\c{}}}1
            {œ}{{\oe{}}}1
            {ù}{{\`u}}1
            {°}{{\up{$\circ$}}}1,  
        stringstyle=\color{green!40!black},
        }
    
    
\newcommand{\ligne}{\noindent \rule{\linewidth}{.5pt}}
\newcommand{\pointilles}{\noindent \rule{0pt}{1cm}\makebox[\linewidth]{\dotfill}}
\newcommand{\interligne}{\rule[-8pt]{0pt}{25pt}}
\newcommand{\reponse}[2]{\noindent \interligne \dotfill}
\newcommand{\minireponse}[1]{\makebox[1cm]{\interligne \dotfill}}

%%%%%%%%%%%%%%%%%%%%%%%%%%%%%%%%%%%%%%%%%%%%%%%%%%%%%%%%%%%%%%%%
\begin{document}

\begin{center}\large Corrigé des exercices.\end{center}

\begin{exo}[Programme de calcul]~
    \lstinputlisting{exo1.py}
\end{exo}

\begin{exo}[perroquet]~
    \lstinputlisting{Exo2020-1.py}
\end{exo}

\begin{exo}[donne moi un prénom]~
    \lstinputlisting{Exo2020-2.py}
\end{exo}

\begin{exo}[puissance]~
    \lstinputlisting{exo2.py}
\end{exo}

\begin{exo}[écho]~
    \lstinputlisting{Exo2020-3.py}
\end{exo}

\begin{exo}[compte à rebours]~
    \lstinputlisting{Exo2020-4.py}
\end{exo}

\begin{exo}[téléchargement]~
    \lstinputlisting{Exo2021-2.py}
\end{exo}

\newpage

\begin{exo}[table de multiplication]~
    \lstinputlisting{exo3.py}
\end{exo}

\begin{exo}[sondage]~
    \lstinputlisting{Exo2020-5.py}
\end{exo}

\begin{exo}[nombre positif]~
    \lstinputlisting{Exo2020-6.py}
\end{exo}

\begin{exo}[Pythagore]~
    \lstinputlisting{exo4.py}
\end{exo}

\begin{exo}[carrés parfaits]~
    \lstinputlisting{exo5.py}
\end{exo}

\newpage

\begin{exo}[PGCD]~
    \lstinputlisting{exo6.py}
\end{exo}

\begin{exo}[plus grand de 3 nombres]~
    \lstinputlisting{exo7.py}
\end{exo}

\begin{exo}[bientôt Noël]~
    \lstinputlisting{Exo2021-1.py}
\end{exo}

\end{document}
